\documentclass[a4paper]{article}

%% Language and font encodings
\usepackage[english]{babel}
\usepackage[utf8x]{inputenc}
\usepackage[T1]{fontenc}

%% Sets page size and margins
\usepackage[a4paper,top=3cm,bottom=2cm,left=3cm,right=3cm,marginparwidth=1.75cm]{geometry}

%% Useful packages
\usepackage{amsmath}
\usepackage{mathtools}
\usepackage{graphicx}
\usepackage[colorinlistoftodos]{todonotes}
\usepackage[colorlinks=true, allcolors=blue]{hyperref}

\title{Rank Optimization}
\author{Lionel Chiron}

\begin{document}
\maketitle

\begin{abstract}
Your abstract.
\end{abstract}

\section{Introduction}

We devise here an analytical solution for the rank optimization for the approximation through random projections.
This solution is then compared to the numerical approach.



\section{Some examples to get started}

\subsection{Fillig equation}

\begin{equation}
      u'= -\frac{u^2}{u^2+v^2}/p
\end{equation}
\begin{equation}
      v'= -\frac{v^2}{u^2+v^2}/n
\end{equation}

we deduce

\begin{equation}
    \frac{u'}{u^2} - \frac{v'}{v^2}= p/n
\end{equation}
\begin{equation}
    p u'+n v' = 1
\end{equation}

it follows

\begin{equation}
    \frac{1}{u} - \frac{1}{v}= \frac{p}{n}t+k
\end{equation}
\begin{equation}
    p u+n v = t+l
\end{equation}

Hence

\begin{equation}
    n(\frac{p}{n}t+k)u^2+((p+n)+(t+l)(\frac{p}{n}t+k))u-(t+l) = 0
\end{equation}

giving the solution


\begin{equation}
    u = \frac{-(l+t)(at+k)-(n+p)-\sqrt{((l+t)(at+k)+n+p)^2+4n(l+t)(at+k)}}{2n(at+k)}
\end{equation}

with $a=p/n$



\subsection{How to add Citations and a References List}

You can upload a \verb|.bib| file containing your BibTeX entries, created with JabRef; or import your \href{https://www.overleaf.com/blog/184}{Mendeley}, CiteULike or Zotero library as a \verb|.bib| file. You can then cite entries from it, like this: \cite{greenwade93}. Just remember to specify a bibliography style, as well as the filename of the \verb|.bib|.

%You can find a \href{https://www.overleaf.com/help/97-how-to-include-a-bibliography-using-bibtex}{video tutorial here} to learn more about BibTeX.

We hope you find Overleaf useful, and please let us know if you have any feedback using the help menu above --- or use the contact form at \url{https://www.overleaf.com/contact}!

\bibliographystyle{alpha}
\bibliography{sample}

\end{document}
